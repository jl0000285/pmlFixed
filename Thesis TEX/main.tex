%%%%%%%%%%%%%%%%%%%%%%%%%%%%%%%%%%%%%%%%%
% Masters/Doctoral Thesis
% LaTeX Template
% Version 2.5 (27/8/17)
%
% This template was downloaded from:
% http://www.LaTeXTemplates.com
%
% Version 2.x major modifications by:
% Vel (vel@latextemplates.com)
%
% This template is based on a template by:
% Steve Gunn (http://users.ecs.soton.ac.uk/srg/softwaretools/document/templates/)
% Sunil Patel (http://www.sunilpatel.co.uk/thesis-template/)
%
% Template license:
% CC BY-NC-SA 3.0 (http://creativecommons.org/licenses/by-nc-sa/3.0/)
%
%%%%%%%%%%%%%%%%%%%%%%%%%%%%%%%%%%%%%%%%%

%----------------------------------------------------------------------------------------
%	PACKAGES AND OTHER DOCUMENT CONFIGURATIONS
%----------------------------------------------------------------------------------------

\documentclass[
11pt, % The default document font size, options: 10pt, 11pt, 12pt
%oneside, % Two side (alternating margins) for binding by default, uncomment to switch to one side
english, % ngerman for German
singlespacing, % Single line spacing, alternatives: onehalfspacing or doublespacing
%draft, % Uncomment to enable draft mode (no pictures, no links, overfull hboxes indicated)
%nolistspacing, % If the document is onehalfspacing or doublespacing, uncomment this to set spacing in lists to single
%liststotoc, % Uncomment to add the list of figures/tables/etc to the table of contents
%toctotoc, % Uncomment to add the main table of contents to the table of contents
%parskip, % Uncomment to add space between paragraphs
%nohyperref, % Uncomment to not load the hyperref package
headsepline, % Uncomment to get a line under the header
%chapterinoneline, % Uncomment to place the chapter title next to the number on one line
%consistentlayout, % Uncomment to change the layout of the declaration, abstract and acknowledgements pages to match the default layout
]{MastersDoctoralThesis} % The class file specifying the document structure

\usepackage[utf8]{inputenc} % Required for inputting international characters
\usepackage[T1]{fontenc} % Output font encoding for international characters

\usepackage{mathpazo} % Use the Palatino font by default

\usepackage[backend=bibtex,style=authoryear,natbib=true]{biblatex} % Use the bibtex backend with the authoryear citation style (which resembles APA)

\addbibresource{example.bib} % The filename of the bibliography

\usepackage[autostyle=true]{csquotes} % Required to generate language-dependent quotes in the bibliography

%----------------------------------------------------------------------------------------
%	MARGIN SETTINGS
%----------------------------------------------------------------------------------------

\geometry{
	paper=a4paper, % Change to letterpaper for US letter
	inner=2.5cm, % Inner margin
	outer=3.8cm, % Outer margin
	bindingoffset=.5cm, % Binding offset
	top=1.5cm, % Top margin
	bottom=1.5cm, % Bottom margin
	%showframe, % Uncomment to show how the type block is set on the page
}

%----------------------------------------------------------------------------------------
%	THESIS INFORMATION
%----------------------------------------------------------------------------------------

\thesistitle{A comparison of metalearning strategies} % Your thesis title, this is used in the title and abstract, print it elsewhere with \ttitle
\supervisor{Dr. Thomas Smolinski \textsc{Smith}} % Your supervisor's name, this is used in the title page, print it elsewhere with \supname
\examiner{} % Your examiner's name, this is not currently used anywhere in the template, print it elsewhere with \examname
\degree{Masters in Science} % Your degree name, this is used in the title page and abstract, print it elsewhere with \degreename
\author{John \textsc{Liddell}} % Your name, this is used in the title page and abstract, print it elsewhere with \authorname
\addresses{} % Your address, this is not currently used anywhere in the template, print it elsewhere with \addressname

\subject{Computer Science} % Your subject area, this is not currently used anywhere in the template, print it elsewhere with \subjectname
\keywords{} % Keywords for your thesis, this is not currently used anywhere in the template, print it elsewhere with \keywordnames
\university{\href{http://www.desu.edu}{Delaware State University}} % Your university's name and URL, this is used in the title page and abstract, print it elsewhere with \univname
\department{\href{http://department.university}{Department of computer science}} % Your department's name and URL, this is used in the title page and abstract, print it elsewhere with \deptname
\group{\href{http://researchgroup.university.com}{CIBIL}} % Your research group's name and URL, this is used in the title page, print it elsewhere with \groupname
\faculty{\href{http://faculty.university.com}{Faculty Name}} % Your faculty's name and URL, this is used in the title page and abstract, print it elsewhere with \facname

\AtBeginDocument{
\hypersetup{pdftitle=\ttitle} % Set the PDF's title to your title
\hypersetup{pdfauthor=\authorname} % Set the PDF's author to your name
\hypersetup{pdfkeywords=\keywordnames} % Set the PDF's keywords to your keywords
}

\begin{document}

\frontmatter % Use roman page numbering style (i, ii, iii, iv...) for the pre-content pages

\pagestyle{plain} % Default to the plain heading style until the thesis style is called for the body content

%----------------------------------------------------------------------------------------
%	TITLE PAGE
%----------------------------------------------------------------------------------------

\begin{titlepage}
\begin{center}

\vspace*{.06\textheight}
{\scshape\LARGE \univname\par}\vspace{1.5cm} % University name
\textsc{\Large Doctoral Thesis}\\[0.5cm] % Thesis type

\HRule \\[0.4cm] % Horizontal line
{\huge \bfseries \ttitle\par}\vspace{0.4cm} % Thesis title
\HRule \\[1.5cm] % Horizontal line

\begin{minipage}[t]{0.4\textwidth}
\begin{flushleft} \large
\emph{Author:}\\
\end{flushleft}
\end{minipage}
\begin{minipage}[t]{0.4\textwidth}
\begin{flushright} \large
\emph{Supervisor:} \\
\end{flushright}
\end{minipage}\\[3cm]

\vfill

\large \textit{A thesis submitted in fulfillment of the requirements\\ for the degree of \degreename}\\[0.3cm] % University requirement text
\textit{in the}\\[0.4cm]
\groupname\\\deptname\\[2cm] % Research group name and department name

\vfill

{\large \today}\\[4cm] % Date
%\includegraphics{Logo} % University/department logo - uncomment to place it

\vfill
\end{center}
\end{titlepage}

%----------------------------------------------------------------------------------------
%	DECLARATION PAGE
%----------------------------------------------------------------------------------------

\begin{declaration}
\addchaptertocentry{\authorshipname} % Add the declaration to the table of contents
\noindent I, \authorname, declare that this thesis titled, \enquote{\ttitle} and the work presented in it are my own. I confirm that:

\begin{itemize}
\item This work was done wholly or mainly while in candidature for a research degree at this University.
\item Where any part of this thesis has previously been submitted for a degree or any other qualification at this University or any other institution, this has been clearly stated.
\item Where I have consulted the published work of others, this is always clearly attributed.
\item Where I have quoted from the work of others, the source is always given. With the exception of such quotations, this thesis is entirely my own work.
\item I have acknowledged all main sources of help.
\item Where the thesis is based on work done by myself jointly with others, I have made clear exactly what was done by others and what I have contributed myself.\\
\end{itemize}

\noindent Signed:\\
\rule[0.5em]{25em}{0.5pt} % This prints a line for the signature

\noindent Date:\\
\rule[0.5em]{25em}{0.5pt} % This prints a line to write the date
\end{declaration}

%----------------------------------------------------------------------------------------
%	ABSTRACT PAGE
%----------------------------------------------------------------------------------------

\begin{abstract}
\addchaptertocentry{\abstractname} % Add the abstract to the table of contents
Determining what algorithm to use when analyzing a dataset is a problem as old as machine learning itself. In some cases, the
individuals wishing to perform an analysis have access to an expert that can simply tell them which algorithm is best in the
given situation. In other situations, the individuals wishing to perform analysis may not have the budget neccessary to
acquire access to such an export, in which case the usage of a metalearner becomes appropriate. With a metalearner one feeds
the metalearner a dataset and it returns to the user what it thinks is the most appropriate machine with which to perform
analysis. To get to the point wherein a decision can be made on new datasets the metalearner itself must first be trained,
and this training itself requires some sort of learning strategy. The comparison of the learning strategies used to train a
metalearner will be the central focus of this project.
The act of utilizing a metalearning strategy generally involves the following activities: one must take a set of datasets,
get a measure of how well those datasets perform with a given body of algorithms then use the information contained within
this "metabase" to run an algorithm that will then spit out a resulting machine for consideration. Here in lies the meat of
the work that must be done to accomplish my stated goal. The system that I will build will first iterate thru the set of
learning strategies to be compared, building metabases for each of them and recording how long it took to build each of
these metabases. It will then take a set of test datasets and have machines spit out for each of these datasets, with the
resulting choice being recorded as the choices are made. A time of traing/accuracy analysis will then be performed between
the different learning strategies
\end{abstract}

%----------------------------------------------------------------------------------------
%	ACKNOWLEDGEMENTS
%----------------------------------------------------------------------------------------

\begin{acknowledgements}
\addchaptertocentry{\acknowledgementname} % Add the acknowledgements to the table of contents
The acknowledgments and the people to thank go here, don't forget to include your project advisor\ldots
\end{acknowledgements}

%----------------------------------------------------------------------------------------
%	LIST OF CONTENTS/FIGURES/TABLES PAGES
%----------------------------------------------------------------------------------------

\tableofcontents % Prints the main table of contents

\listoffigures % Prints the list of figures

\listoftables % Prints the list of tables

%----------------------------------------------------------------------------------------
%	ABBREVIATIONS
%----------------------------------------------------------------------------------------

\begin{abbreviations}{ll} % Include a list of abbreviations (a table of two columns)

\textbf{LAH} & \textbf{L}ist \textbf{A}bbreviations \textbf{H}ere\\
\textbf{WSF} & \textbf{W}hat (it) \textbf{S}tands \textbf{F}or\\

\end{abbreviations}

%----------------------------------------------------------------------------------------
%	PHYSICAL CONSTANTS/OTHER DEFINITIONS
%----------------------------------------------------------------------------------------

\begin{constants}{lr@{${}={}$}l} % The list of physical constants is a three column table

% The \SI{}{} command is provided by the siunitx package, see its documentation for instructions on how to use it

Speed of Light & $c_{0}$ & \SI{2.99792458e8}{\meter\per\second} (exact)\\
%Constant Name & $Symbol$ & $Constant Value$ with units\\

\end{constants}

%----------------------------------------------------------------------------------------
%	SYMBOLS
%----------------------------------------------------------------------------------------

\begin{symbols}{lll} % Include a list of Symbols (a three column table)

$a$ & distance & \si{\meter} \\
$P$ & power & \si{\watt} (\si{\joule\per\second}) \\
%Symbol & Name & Unit \\

\addlinespace % Gap to separate the Roman symbols from the Greek

$\omega$ & angular frequency & \si{\radian} \\

\end{symbols}

%----------------------------------------------------------------------------------------
%	DEDICATION
%----------------------------------------------------------------------------------------

\dedicatory{For/Dedicated to/To my\ldots}

%----------------------------------------------------------------------------------------
%	THESIS CONTENT - CHAPTERS
%----------------------------------------------------------------------------------------

\mainmatter % Begin numeric (1,2,3...) page numbering

\pagestyle{thesis} % Return the page headers back to the "thesis" style

% Include the chapters of the thesis as separate files from the Chapters folder
% Uncomment the lines as you write the chapters

%Chapter 1
\chapter{Introduction}
\label{Introduction}
Determining what algorithm to use when analyzing a dataset is a problem as
old as machine learning itself. In ``No free lunch theorems for optimization''
Wolpert and Macready demonstrate that the relative performance of any two given
machine learning algorithms will be uniform across all datasets, that is to say
a machine learning algorithms performance is contingent on the problem space in
which the algorithm is operating. As such, the decision of algorithm is not
arbitrary and some strategy must be employed in order to decide
on an algorithm. In some cases, the individuals wishing to perform an analysis
have access to an expert, possibly themselves, that can simply tell them which
algorithm is best in the given situation. In other situations, the individuals
wishing to perform analysis may not have the budget necessary to acquire access
to such an expert, in which case the usage of a meta-learner becomes appropriate.
For instance, at Walmart Labs, meta learning algorithms are used to decide how
best to detect placeholders and to identify fraudulent transactions, all without
the use of manual parameter tunning or even direct algorithm selection \cite{Gupta}.
With a meta-learner, one feeds the meta-learner a dataset, and it returns to the
user what it thinks is the most appropriate machine with which to perform
analysis. To get to the point wherein a decision can be made on new datasets, the
meta-learner itself must first be trained, and this training requires some
sort of learning strategy. This fact suggests that the decision of what
meta-learning strategy to use for some given body of datasets should be
susceptible to the previously mentioned no free lunch theorem, that is to say
that some meta-learning strategies will work better on some given set of
databases than others. The confirmation or denial of this theorem in this
context is the goal of this thesis. Including the current one, this thesis is
comprised of five chapters. In Chapter 2, a review of the base machine and meta
learning strategies used within the experiment is done. Chapter 3 describes the
structure of the experiments code at a high level. Chapter 4 analyzes the
results table in order to determine whether or not one meta-learning strategy
strictly dominates. Chapter 5 presents a summary of the document and addresses a
possible weakness in the results.

\pagebreak

%%Chapter 2
\chapter{Review of the literature}
\label{Chapter2}
%---------------------------------------------------------------------------------------
\section{No Free Lunch Theorem}
Introduced in Wolpert and Macready's ``No Free Lunch Theorems for Optimization''
1997 paper, the No Free Lunch theorem states that the performance of all
algorithms, when averaged out across all datasets, should be the same; that is
to say there is no one algorithm that is universally the best. The root cause of
this phenomenon is in that differing algorithms make different assumptions
about the distributions from which the data the algorithms work with arises. A
learning algorithm with an implicit assumption of a random distribution will
have a far lower test case classification accuracy than an algorithm that
assumes a Gaussian distribution if the distribution from which the set of
observed samples derives is truly normal and vice versa, if the
distribution is truly random, the Gaussian classifier's accuracy will suffer
relative to the classifier with a random assumption.
%-------------------------------------------------------------------------------------------
\subsection{Brute Force Metabase}
The most basic meta-machine learning algorithm. The accuracies of the
meta-learners producible machines for some metabase are gathered. To classify
a new dataset $d_n$, a clustering algorithm (K-Means in the case of this experiment) is
used to find $d_m$, the dataset within the metabase with which the new dataset $d_n$ is
most similar. The algorithm which had the greatest classification accuracy for
the metabase dataset $d_m$ will then be returned by the meta-learner.
%-------------------------------------------------------------------------------------------
\subsection{Active Meta Learning}
The second of the meta-learning strategies implemented within the study, Active
Meta Learning is a meta learning technique  ``that reduces the cost of
generating Meta examples by selecting relevant Meta examples'' \cite{Bhatt}.
What this entails is a decision on what datasets to allow into a meta-learner's
metabase. Rather than analyze every candidate meta base dataset, an active
meta-learner will analyze the next dataset with the highest uncertainty. The
relative uncertainty between two datasets is defined to be:
$$\delta(V_x,d_i,V_x,d_j) = \frac{|V_x,d_i - V_x,d_j|}{Max_{k\neq i}(V_x,d_k)- Min_{k\neq i}(V_x,d_k)}$$
where $V_x,d_k$ is the value of some metaparameter $V_x$ for dataset $d_k$,
$Max_{k\neq i}(V_x,d_k)$ is the maximum value of $V_x,d_k$ when dataset $i$ is
removed and $Min_{k\neq i}(V_x,d_k)$ is its corresponding minimum. Determining
which dataset has the overall highest uncertainty can be done via the following
procedure. First, sum the relative uncertainties for each dataset and
meta-parameter combination. Then, rank the uncertainty scores of the datasets within
each meta-parameter. After obtaining the uncertainty ranks within each parameter
for each dataset, sum the parameter ranks in order to obtain an overall
uncertainty rank for each dataset. Finally, select the parameter with the
highest rank for inclusion in the metabase. The equation representing the
overall uncertainty score in a specific metaparameter $V_x$ for dataset $d_i$ is
$$\delta(V_x,d_i) = \frac{\sum_{j\neq i} |V_x,d_i - V_x,d_j|}{Max_{k\neq i}(V_x,d_k)- Min_{k\neq i}(V_x,d_k)}$$
%-----------------------------------------------------------------------------------------------
\subsection{Predicting Relative Performance of Classifiers from Samples}
The third of the meta-learning strategies implemented within this study is one in
which a representative subsection of the metabase is trained with each algorithm
entirely, after which point the rest of the metabase undergoes curve
sampling analysis; the accuracies of the base algorithms are predicted from
run curve similarity rather than directly ran \cite{Leite}.
As with the other two meta learning strategies, the label for new datsets is
then determined via clustering with the datasets contained within the metabase.
%----------------------------------------------------------------------------------------
\section{Summary of Producible Machines}
The strategies mentioned in the previous section all consume a vector
representation of some dataset, and then make a guess as to what algorithm would best
be able to classify its data. The machines that these strategies can choose from are
the K-means clustering algorithm, a neural network, a naive Bayes classifier, the support
vector machine, and regression; with the results coming from the regression
machine being cast into classificatory bins from the real valued result that it
would produce. An in depth description of each of these different learning
algorithms will comprise the rest of this chapter.
%--------------------------------------------------------------------------------------------
\section{Linear Regression}
Linear regression is one of the most common and oldest machine learning
techniques. It asserts that the response is a linear function
of the inputs \cite{Murphy}. This relation takes the following form:
$$ y(\textbf{x}) = \textbf{w}^T\textbf{x} + \epsilon = \sum_{j=1}^{D}w_jx_j + \epsilon $$
where $w^Tx$ represents the inner or scalar product between the input vector $x$
and the model's weight vector $w^T$, and $\epsilon$ is the residual error
between our linear predictions and the true response.

To fit a linear regression model, the least squares approach is usually used.
Given some  ``overdetermined'' linear system (that is to say a system in which
there are more data points than parameters), one can write an expression for the
sum of squares of the system
$$S(\beta) = (y_1 - \beta x_1)^2 + (y_2 - \beta x_2)^2 + ... (y_n - \beta x_n)^n$$
and then take the partial derivative of this sum of squares deviations with respect
to each of the components of $\beta$, set them to zero, then solve the resulting
equations to directly determine the values of the parameters that
minimize the sum of the squared errors of the system. With linear regression in
two dimensions (one dimension in the Independent variable and one dimension in
the Dependent variable, we see a system with two parameters
$\beta_0 = y intercept$ and $\beta_1 = slope$. If we had, for example, 3 data
points (2,1),(3,7), and (4,5) we would have the equations
$\beta_0 + 2*\beta_1 = 1$, $\beta_0 + 3*\beta_1 = 7$, and
$\beta_0 + 4*\beta_1 = 5$. The sum of squared errors would then be
 $S(\beta_0,\beta_1)= [1 - (\beta_0 + 2*\beta_1)]^2 + [7 - (\beta_0 + 3*\beta_1)]^2 + [5 - (\beta_0 + 4*\beta_1)]^2$
,which we could then differentiate with respect to $\beta_0$ and $\beta_1$ then
directly solve the resulting set of linear equations directly for the minimum of
the summed squares.
%--------------------------------------------------------------------------------------------
\section{Naive Bayes}
The Naive Bayes classifier algorithm fits a set of data to Bayes' Theorem with a
strong assumption of feature independence. Given a set of discrete-valued
features $x \in {1,...,K}^D$, we can calculate the class conditional density for
each feature, then, with our assumption of independence, generate a guess
at what the class should be for a new input by multiplying the conditional
likelihood values for each of the new inputs features times the prior on the
desired to be known class, that is to say
$p(y|\textbf{x}) \propto p(y) \sum_{j}^{D}p(x_j|y)$.
The calculation of the posterior probability for a new example can be done
manually, or can be derived from distributions that are inferred from the
provided data. Consider, for example, a collection of data listing individuals
that did or did not purchase a house from a real estate agent, where, for some
reason or another, the only data remaining pertaining to these individuals is
what their income level was, what their age was, and how far they have to or
would have had to drive to work from their new home.

Say we get a new datapoint: income: \$25,000, age: 30, distance: 10. In this
case the conditional likelihood of this data given a yes for each of the
individual features is 2/9, 1/9, and 2/9 respectively. The prior on yes is 1/3.
The marginal likelihoods of each the individual features are 2/9, 1/9 and 2/9
respectively. As such, the posteriors for our new datapoint are
$p(y=yes|x)=\frac{(2/9)*(1/9)*(2/9)*(9/27)}{(6/27)*(3/27)*(6/27)} = \frac{0.00182}{0.00548} = 0.33$ and $p(y=no|x)=\frac{(4/18)(2/18)(4/18)(18/27)}{(6/27)(3/27)(6/27)} = \frac{0.0036}{0.00548} = 0.66$.

Note that $0.66 > 0.33$ and as such our classifier would label this datapoint
with a no, this individual is not likely to purchase a house.

%--------------------------------------------------------------------------------------------
\section{Support Vector Machine}
The support vector machine (svm) is a two-group classification algorithm that attempts
to find a hyperplane that separates the inputs within a given input space
with a maximum margin of separation between the hyperplane and the
``support vectors,'' those vectors on either side of the hyperplane
that are closest to it. To arrive at a form of the support vector machine that
can be used to classify new inputs, one first needs a representation of the
potential separating hyperplane $$y_i(\mathbf{w}*\mathbf{x_i} + b) > 1$$ where
$y_i$ is the truth label of given training input $x_i$, $w$ is a vector
normal to our candidate separating hyperplane that represents how much
``weight'' is to be applied to an input, and $b$ is a bias constant representing
the threshold the weight/input product needs to pass before it is considered
classified. The distance of a given hyperplane can be determined by calculating
the difference between these previously mentioned ``support vectors'' in the
direction normal to this hyperplane. This difference can be calculated via the
following equation:
$$(\bold{x_{s+}}-\bold{x_{s-}})\bold{\frac{w}{||w||}}$$ where $\bold{x_{s+}}$
and $\bold{x_{s-}}$ are respectively positive and negative support vectors and
$\bold{\frac{w}{||w||}}$ is the unit vector in the direction normal to the hyper
plane towards the positive examples. The size of the margin is $2/||w||$. As
such, the discovery of a working svm can be accomplished by solving a constrained
optimization problem in which the thing to be minimized is $1/2||w||^2$, subject
to the constraints $y_i(\bold{w}*\bold{x_i} + b) = 1$ for support vectors.
Crafting an expression of this constrained optimization that can be solved by a
computer can be done by taking the Lagrangian:
$$L(\bold{W},\bold{\Lambda}, \bold{Y}) = 1/2||\bold{w}||^2 \sum_{i=1}^{n}y_i(\bold{w}*\bold{x_i}+b)-1)$$
then taking care of the fact that the vector $w_o$ that determines the optimal
hyperplane can be written as a linear combination of the training vectors:
$w_{0}=\sum_{i=1}^{n}y_{i}\alpha_{i}^{0}\bold{x_i}$ \cite{Vapnik}. Swapping this
equation into the Lagrangian yields:
$$L = \sum \alpha_{i} - 1/2 \sum_{i}\sum_{j}\alpha_{i}\alpha_{j}y_{i}y_{j}(\bold{x_i}\cdot\bold{x_j})$$
at which point one should consult their closest computer so that it can maximize
this expression.

This final form of the Lagrangian reveals the support vector machines most
powerful attribute: the kernel. The optimization of the hyperplane within the
inputs depends only on their dot product of pairs of inputs; they do not appear
anywhere else in the Lagrangian other than the very end and then only so as
pairs of dot products. This fact allows the writing of a decision function on
new inputs: $$f(\bold{u}) = \sum_{i}^{N}\alpha_{i}y_{i}(\bold{x_i}\cdot\bold{u})+b$$
where $\bold{u}$ is a vector whose label we do not know. The support vector
machine can use kernels to map input vectors into
non-linear high-dimensional feature space without actually calculating the
position of the vectors within that feature space \cite{Vapnik}. The kernel
accomplishes this by calculating the distance between (or similarity) of its two
input vectors in this space without reference to their exact position within
this higher space. This then allows the computation of a linear separation
between the points in this higher dimensional space which translates into a
non-linear separation for the vectors in their original lower dimensional space
where a separation might otherwise not have been discoverable.
%--------------------------------------------------------------------------------------------
\section{K-Means Clustering}
The objective of the k-means algorithm is to partition a dataset into k groups
such that the points within some group are all closest to
the mean of that group than they are to any other group. A clear
informal explanation of the work that the k-means algorithm performs
was given by James MacQueen in 1967: ``...the k-means procedure
consists of simply starting with k groups each of which consists of a
single random point, and thereafter adding each new point to the
group whose mean the new point is nearest. After a point is added to
a group, the mean of that group is adjusted in order to take account
of the new point. Thus at each stage the k-means are, in fact, the
means of the groups they represent'' \cite{MacQueen}. Formally stated,
given an integer $k$ and a set of $n$ data points in
$\mathbb{R}^{d}$ the K-means algorithm seeks to minimize  $\Phi$, the
over all total summed in class distance between each point and its
closest center such that $\mathbb \Phi = \sum_{x \in X} min_{c \in C}{x-c^{2}}$
\cite{Arthur}.

The k-means model is a type of Gaussian mixture model that is trained with a procedure
called expectation maximization. Given a set of distributions with missing data, mixture models tend to have derivatives that are either difficult to define
or are entirely undefinable. On the other hand, the calculation of some ML/MAP (maximum likelihood/maximum a posteriori)
estimates for some set of models can generally be calculated with little
difficulty if every point within the distributions is known (at which point our
learner would obviously have nothing to do) and thus calculus would be entirely
unnecessary ($i.e.$, it would not matter that the derivative cannot be defined).
Expectation maximization uses this fact in order to obtain an estimation of the
ML/MAP indirectly. The algorithm consists of two steps. First, an estimate as
to what the expected value of the hidden data is based off the current guess for the
parameters is made. Then the likelihood function for the parameters is maximized under
the assumption that the data discovered in the previous step is complete, $i.e.$, that there
is no longer any hidden data. These steps are then repeated until some convergence criteria
is met. The k-means is exactly this type of algorithm, but with the covariance matrix
$\Sigma_{k} = \rho^{2}*I_{D}$ and the mixing weights $\Pi_{k} = 1/K$ all being fixed, such
that the only free parameters are the cluster centers $\mu_{k} \in \mathbb{R}^{D}$,
and such that the hidden data that is the ground truth label of the data points.
%--------------------------------------------------------------------------------------------
\section{Neural Networks}
A neural network is a type of machine learning algorithm that mimics
the interconnectivity of animal brains in order to automatically
discover rules to classify given inputs. The neural network is one of the most
flexible learning algorithms within literature, so flexible in fact that it is
capable of approximating any continuous function \cite{Hornik}. As such, its
inclusion within a metalearning system is almost mandatory.  Genrally,
a neural network system works by first being presented with a set of classified or
unclassified inputs. Said system will then attempt to
make a decision on these inputs on which an error value will then be
assigned. The system will then see some kind of correction function
applied to it. This process will continue until the system has
exhausted its supply of training data, at which point it will
hopefully have discovered a strong set of rules for peforming whatever
work it is that it was designed to perform.

The type of neural network that will be used within this thesis is
what is called the feed-forward neural network (multilayer perceptron,
a.k.a. MLP). The feed forward neural network is essentially a series of
logistic regression models stacked on top of each other, with the
final layer being either another logicstic regression or linear
regression model depending on whether or not a classification or
regression problem is being solved \cite{Murphy}. The leftmost
layer of this stack is called the input layer and consists of a set of
neurons ${x_i|x_1,x_2,x_3,...,x_m}$ representing the input's
features. Each neuron in the hidden layer transforms the values from
the previous layer via weighted linear summation $w_1X_1 + w_2x_2 +...+w_mx_m$
which is then passed into a non linear-action function $g()$, such as the
logistic function or the hyperbolic tangent function. It is important to note
that $g$ must be non-linear, otherwise the entire model will collapse into a
large linear regression model of the form $y = w^T(Vx)$ \cite{Murphy}.

The multi-layer perceptrons created in this experiment will be trained used in
by an error propagation/training technique called backpropagation.
Backpropagation is a procedure that repeatedly adjusts the weights of the
connections in a neural network so as to minimize a measure of the difference
between the actual output vector of the net and the desired output
vector \cite{Rumelhart}. In order to accomplish this, the algorithm
adjusts the weights of the nerual network by considering the error of
the outputs then minimizes this error via gradient decent with respect
to each of the weights within the network. Specifically, the gradient
vector of the negative log likelihood error on the output neurons is
computed by use of the chain rule of calculus.\cite{Murphy}. Say we
have a one layer neural network in which the hidden layer is described
by $\alpha^{L} = \sigma(w^{L}\alpha^{L-1} + b^{L}) = \sigma(z^{L})$,
where $L$ superscript refers to the hidden layer and $L-1$ refers
to the input layer of the network. The parameters of this network can
be said to be $\Theta = (V,W)$ where $V$ is the weight vector for the input
layer and $W$ is the weight vector for the hidden layer. The error (or
more specifically the costs function) of a such a network is given by:
$$J(\Theta ) = - \sum_n\sum_k(\hat{y}_{nk}(\Theta)-y_{nk})^2$$
in the case of regression, and via cross entropy
$$J(\Theta ) = - \sum_n\sum_ky_{nk}log\hat{y}_{nk}(\Theta)$$
in the case of classification. The gradient of this error $\nabla_{\Theta}J$
is found via the chain rule of calculus:
$$\frac{\partial C}{\partial
w^{L}} = \frac{\partial z^{L}}{\partial
w^{L}}\frac{\partial \alpha}{\partial z^{L}}\frac{\partial
  C}{\partial \alpha^{L}}$$
This equation is easiest to understand if read from right to left.
Notice how in each stage the rates being compared are between nearest
elements, first the error to the output that produced it, then the output
to the element to which the non-linearity is applied, then finally the the
non-linerity recieving value to the weight vector. The result of this
calculation easily gives us the direction of the gradient, the negative of
which we will use to modify $w^{L}$ in a direction that will reduce the output
error. Reduction of the error of a multilayer perceptron with more
than one neuron in each layer works mostly the same way. Once again,
the chain rule of calculus is used in order to find the derivative of
the output error with respect to the weights of the connections
between the hidden layer before the output neurons and the output
neurons $\frac{\partial C}{\partial w^{L}_{n-j}}$ where $L$ is the
target neurons layer (the last layer in this case), $n$ is the index of
a neuron within this layer, and $j$ is the index of the neuron in the
previous layer $L-1$ from which neuron $n$ is recieving input (note
that in this case the hyphen does not mean subtract, but rather
indicates that there is a connection between these neurons). For an output
neuron, its error is then given by:
$$\frac{\partial C}{\partial w^{L}_{n-j}} = \frac{\partial z^{L}_{j}}{\partial w^{L}_{n-j}}\frac{\partial \alpha^{L}_{j}}{\partial z^{L}_{j}}\frac{\partial C}{\partial \alpha^{L}_{j}}$$
with the error relative to neurons earlier in the network being calculable by continuing usage of the chain rule.

%For an excellent and intuitive explanation of how neural networks work, pls consider viewing the animated overview of the method at 3Blue1Brown's youtube channel.
% WAY TOO INFORMAL!

%%Chapter 3
\chapter{Research Design}
\label{Chapter3}
%-------------------------------------------------
\section{General Plan}
The overall goal of this experiment is to determine whether one meta learning
strategy within a given set can strictly dominate the others. The core elements
required in order to determine this are: (1) a set of meta learning strategies to be
compared, (2) sets of metabase datasets on which to apply the different strategies,
and (3) a means to evaluate the results so as to determine the relative performance
of the meta learners. The general flow of the program that constitutes this
experiment begins with a set of unprocessed datasets, then extracts the
meta features that are required to perform dataset clustering and to
run the active meta learning strategy. The program then constructs 10
meta learning bases with the elements in these sets chosen at random. A run with
every machine algorithm and dataset combination is then performed, with the
results being stored in the experiments database. Learning curves for each
dataset/algorithm combination are then crafted. Finally, enough information now
exists within the database to run the meta learning strategies
and extract results. This process is then repeated 30 times, with each run
constituting an individual sample. This produces enough information in order
to perform statistical analysis techniques to test the null hypothesis and
obtain a margin of error of 5 percent. A detailed explanation of the experiment
steps follows now, with a description of the statistical analysis to follow in
Chapter 4.

\begin{figure}
  \includegraphics{Chapters/Images/MetaLearnerFLowchart/MetaLearnerFlowchart.pdf}
  \caption{A visual representation of this experiment's program flow}
\end{figure}

\section{Data Parsing}
The data used in this experiment come from the UCI Irvine Machine learning
repository, the obtainment of which was accomplished via the use of a bash
shell script that allowed the downloading of every dataset in the repository
all at once. To make use of a dataset from the repository, the algorithms used
in this experiment required a vector representation of the dataset currently
being analyzed; the data could not be used without first translating it into
this form. As such, I needed to do two things with the data before making use
of them: ensure that the data could be parsed into a vector via the use of a program,
then write a program in order to do this. The strategy I employed in order to
accomplish these goals was twofold: I first went through the set of candidate
datasets and ensured that non of them were in a format so exotic that they could not
be parsed programmatically. Manual examination of the files revealed that those
of either the .data, .svm, or .dat format were agreeable to formatting and so it
is these that were processed by the parser. These files were then inspected by the
parser, with each dataset's column vectors being inspected one by one. Those
column vectors containing only numerical data were left as is, those with any
non-numerical data were assumed to be categorical, with the categories of said
vector being translated to numbers with a unique number being assigned to each
unique string. Rather than storing the numerical representation of each dataset
within the database, the parsed form of a given dataset is crafted when it is
needed, saving an enormous amount of disk space.

\section{Meta Feature Extraction}
The existence of a parser allows us to craft the first table needed for this
experiment which is one containing the meta features of those datasets that are
parsable. Since the datasets used within this project have vastly
differing structures with respect to metrics such as the number of features
and the maximum and minimum values of these features, the project requires a set
of normalized meta features that are applicable to any possible individual
distribution or set of probability distribution(s). A set of features that meet
these criteria are weighted mean, coefficient of variation, skewness, kurtosis, and
entropy. The vector that represents a given dataset is crafted by taking the
value of each of these attributes for each of said datasets features then
normalizing them by dividing by the total number of features within that dataset,
that is to say
$$F_{ad} = \frac{\sum_{c=i}^{N}f_{ai}}{N}$$
is the meta feature value $a$ for dataset $d$, $c$ is an iterator across columns
for dataset $d$, $f$ is the value of meta feature $F$ applied to individual
column $i$, and $N$ is the number of columns within dataset $d$. The vector that
represents a given dataset is then determined to be
$V_d = (F_{1d}, F_{2d},..., F_{ad})$.

\section{Metabase Construction and Performance Testing}
The work of a meta learning algorithm is essentially the applying the things
known from a specific set of datasets towards a new, unlabeled dataset. That
initial set of datasets is known as a meta database, which I shorten to metabase
for convenience. The core premise of this experiment is the determination as to
whether one meta learning strategy may or may not dominate other
meta learning strategies across a set of different metabases. As such, this
experiment requires multiple sets of metabases in order to produce samples
that can be used to test our hypothesis. The datasets in a given metabase are
randomly chosen from the set of all datasets stored in the database. There
are 10 of these per sample, each being a fifth the size of the entire set of
datasets. Testing the performance of a meta learner is done by using the
meta learning strategy with some given metabase and applying it to every other
dataset within the set of datasets. The guesses a given meta learning strategy
makes with some given metabase are then stored within a database table for later
analysis.

\section{Compiling Results}
Once the guess tables are populated, it is finally possible to compile a table
of results. Each entry in the table notes the meta learning strategy being
evaluated, the metabase collection ($i.e.,$ the sample), the metabase within that
sample that the strategy used in order to analyze its test datasets, and the
accuracy, training time, and rate correct score as its performance metrics,
where the rate correct time measures how often the metalearner makes the correct
guess given the time spent to train it in units of correct guesses per second.

%%Chapter 4
\chapter{Research Findings}
\label{Chapter4}
\section{Run results and Analysis tools}
In order to test the null hypothesis, 30 such samples of the kind described
in chapter 3 were collected. The samples and their means can be seen in tables
4.1 and 4.2 below.

\begin{table}
\begin{tabular}{lrrrrrrrrr}
\toprule
algorithms & \multicolumn{3}{l}{GuessesActive} & \multicolumn{3}{l}{GuessesEx} & \multicolumn{3}{l}{GuessesSamp} \\
positions &             First &  Second &  Third &         First &  Second  &  Third &           First &  Second &  Third \\
\midrule
0  &             1 &  4 &  5 &         6 &  2 &  2 &           3 &  4 &  3 \\
1  &             1 &  4 &  5 &         5 &  2 &  3 &           4 &  4 &  2 \\
2  &             1 &  3 &  6 &         7 &  3 &  0 &           2 &  4 &  4 \\
3  &             1 &  5 &  4 &         6 &  3 &  1 &           3 &  2 &  5 \\
4  &             0 &  6 &  4 &         8 &  2 &  0 &           2 &  2 &  6 \\
5  &             3 &  3 &  4 &         5 &  4 &  1 &           2 &  3 &  5 \\
6  &             4 &  3 &  3 &         4 &  4 &  2 &           2 &  3 &  5 \\
7  &             2 &  3 &  5 &         7 &  2 &  1 &           1 &  5 &  4 \\
8  &             1 &  3 &  6 &         3 &  5 &  2 &           6 &  2 &  2 \\
9  &             0 &  4 &  6 &         7 &  3 &  0 &           3 &  3 &  4 \\
10 &             0 &  6 &  4 &         7 &  3 &  0 &           3 &  1 &  6 \\
11 &             1 &  5 &  4 &         7 &  2 &  1 &           2 &  3 &  5 \\
12 &             3 &  3 &  4 &         5 &  4 &  1 &           2 &  3 &  5 \\
13 &             2 &  5 &  3 &         6 &  3 &  1 &           2 &  2 &  6 \\
14 &             2 &  1 &  7 &         4 &  6 &  0 &           4 &  3 &  3 \\
15 &             1 &  5 &  4 &         6 &  0 &  4 &           3 &  5 &  2 \\
16 &             1 &  4 &  5 &         6 &  4 &  0 &           3 &  2 &  5 \\
17 &             1 &  3 &  6 &         8 &  1 &  1 &           1 &  6 &  3 \\
18 &             1 &  4 &  5 &         7 &  3 &  0 &           2 &  3 &  5 \\
19 &             2 &  4 &  4 &         6 &  2 &  2 &           2 &  4 &  4 \\
20 &             1 &  2 &  7 &         4 &  6 &  0 &           5 &  2 &  3 \\
21 &             3 &  3 &  4 &         2 &  7 &  1 &           5 &  0 &  5 \\
22 &             3 &  4 &  3 &         6 &  4 &  0 &           1 &  2 &  7 \\
23 &             3 &  3 &  4 &         4 &  4 &  2 &           3 &  3 &  4 \\
24 &             2 &  6 &  2 &         7 &  3 &  0 &           1 &  1 &  8 \\
25 &             1 &  3 &  6 &         6 &  2 &  2 &           3 &  5 &  2 \\
26 &             7 &  2 &  1 &         3 &  5 &  2 &           0 &  3 &  7 \\
27 &             0 &  5 &  5 &         7 &  2 &  1 &           3 &  3 &  4 \\
28 &             1 &  2 &  7 &         4 &  5 &  1 &           5 &  3 &  2 \\
29 &             2 &  6 &  2 &         4 &  3 &  3 &           4 &  1 &  5 \\
\bottomrule
\end{tabular}
\caption{Placement results}
\end{table}


\begin{table}
\begin{tabular}{lrrr}
\toprule
{} &  GuessesActive &  GuessesEx &  GuessesSamp \\
\midrule
First &       1.700000 &        3.8 &     4.500000 \\
Second &       5.566667 &        3.3 &     1.133333 \\
Third &       2.733333 &        2.9 &     4.366667 \\
\bottomrule
\end{tabular}
\caption{Placement results means}
\end{table}

If each of the algorithms were truly equal, we would expect the averaged numbers
for each of the positions to be near 3.3. Instead, it appears that the sampler
performed the best. Whether or not these results fall far enough outside
expectation in order to reject the null hypothesis requires the machinary of
classical statistics. Two fairly reliable measures of how unlikely these results
are are the sampling distribution probabilities and t scores of each of the
results. A brief description of each of these statistical methods follows.

\subsection{Exact Sampling Distribution}
The following description losely follows the procedure described in \cite{Cohen}.
In it, the author asks you to imagine testing a coin to see whether or not it
is fair, flipping the coin 1,2,..N times. He then asks you to consider whether
some proportion of heads is actually fair from 0/N, 1/N.., N/N heads. The
propability that some proportion of heads p = i/N is fair can be calculated
exactly with the binomial distribution $$\frac{N!}{i!(N-i)!}r^{i}(1-r)^{N-i}$$.
This situation is analogous to the number of first, second, or third place
finishes some meta-algorithm obtained in this thesis experiment. The probabilty
of proportions for each of the meta-learning algorithms can be seen in table 4.3.

\begin{table}
\begin{tabular}{lrrrrrrrrr}
\toprule
algorithms & \multicolumn{3}{l}{GuessesActive} & \multicolumn{3}{l}{GuessesEx} & \multicolumn{3}{l}{GuessesSamp} \\
positions &             0 &         1 &         2 &         0 &         1 &         2 &           0 &         1 &         2 \\
\midrule
0  &      0.086708 &  0.227608 &  0.136565 &  0.056902 &  0.195092 &  0.195092 &    0.260123 &  0.227608 &  0.260123 \\
1  &      0.086708 &  0.227608 &  0.136565 &  0.136565 &  0.195092 &  0.260123 &    0.227608 &  0.227608 &  0.195092 \\
2  &      0.086708 &  0.260123 &  0.056902 &  0.016258 &  0.260123 &  0.017342 &    0.195092 &  0.227608 &  0.227608 \\
3  &      0.086708 &  0.136565 &  0.227608 &  0.056902 &  0.260123 &  0.086708 &    0.260123 &  0.195092 &  0.136565 \\
4  &      0.017342 &  0.056902 &  0.227608 &  0.003048 &  0.195092 &  0.017342 &    0.195092 &  0.195092 &  0.056902 \\
5  &      0.260123 &  0.260123 &  0.227608 &  0.136565 &  0.227608 &  0.086708 &    0.195092 &  0.260123 &  0.136565 \\
6  &      0.227608 &  0.260123 &  0.260123 &  0.227608 &  0.227608 &  0.195092 &    0.195092 &  0.260123 &  0.136565 \\
7  &      0.195092 &  0.260123 &  0.136565 &  0.016258 &  0.195092 &  0.086708 &    0.086708 &  0.136565 &  0.227608 \\
8  &      0.086708 &  0.260123 &  0.056902 &  0.260123 &  0.136565 &  0.195092 &    0.056902 &  0.195092 &  0.195092 \\
9  &      0.017342 &  0.227608 &  0.056902 &  0.016258 &  0.260123 &  0.017342 &    0.260123 &  0.260123 &  0.227608 \\
10 &      0.017342 &  0.056902 &  0.227608 &  0.016258 &  0.260123 &  0.017342 &    0.260123 &  0.086708 &  0.056902 \\
11 &      0.086708 &  0.136565 &  0.227608 &  0.016258 &  0.195092 &  0.086708 &    0.195092 &  0.260123 &  0.136565 \\
12 &      0.260123 &  0.260123 &  0.227608 &  0.136565 &  0.227608 &  0.086708 &    0.195092 &  0.260123 &  0.136565 \\
13 &      0.195092 &  0.136565 &  0.260123 &  0.056902 &  0.260123 &  0.086708 &    0.195092 &  0.195092 &  0.056902 \\
14 &      0.195092 &  0.086708 &  0.016258 &  0.227608 &  0.056902 &  0.017342 &    0.227608 &  0.260123 &  0.260123 \\
15 &      0.086708 &  0.136565 &  0.227608 &  0.056902 &  0.017342 &  0.227608 &    0.260123 &  0.136565 &  0.195092 \\
16 &      0.086708 &  0.227608 &  0.136565 &  0.056902 &  0.227608 &  0.017342 &    0.260123 &  0.195092 &  0.136565 \\
17 &      0.086708 &  0.260123 &  0.056902 &  0.003048 &  0.086708 &  0.086708 &    0.086708 &  0.056902 &  0.260123 \\
18 &      0.086708 &  0.227608 &  0.136565 &  0.016258 &  0.260123 &  0.017342 &    0.195092 &  0.260123 &  0.136565 \\
19 &      0.195092 &  0.227608 &  0.227608 &  0.056902 &  0.195092 &  0.195092 &    0.195092 &  0.227608 &  0.227608 \\
20 &      0.086708 &  0.195092 &  0.016258 &  0.227608 &  0.056902 &  0.017342 &    0.136565 &  0.195092 &  0.260123 \\
21 &      0.260123 &  0.260123 &  0.227608 &  0.195092 &  0.016258 &  0.086708 &    0.136565 &  0.017342 &  0.136565 \\
22 &      0.260123 &  0.227608 &  0.260123 &  0.056902 &  0.227608 &  0.017342 &    0.086708 &  0.195092 &  0.016258 \\
23 &      0.260123 &  0.260123 &  0.227608 &  0.227608 &  0.227608 &  0.195092 &    0.260123 &  0.260123 &  0.227608 \\
24 &      0.195092 &  0.056902 &  0.195092 &  0.016258 &  0.260123 &  0.017342 &    0.086708 &  0.086708 &  0.003048 \\
25 &      0.086708 &  0.260123 &  0.056902 &  0.056902 &  0.195092 &  0.195092 &    0.260123 &  0.136565 &  0.195092 \\
26 &      0.016258 &  0.195092 &  0.086708 &  0.260123 &  0.136565 &  0.195092 &    0.017342 &  0.260123 &  0.016258 \\
27 &      0.017342 &  0.136565 &  0.136565 &  0.016258 &  0.195092 &  0.086708 &    0.260123 &  0.260123 &  0.227608 \\
28 &      0.086708 &  0.195092 &  0.016258 &  0.227608 &  0.136565 &  0.086708 &    0.136565 &  0.260123 &  0.195092 \\
29 &      0.195092 &  0.056902 &  0.195092 &  0.227608 &  0.260123 &  0.260123 &    0.227608 &  0.086708 &  0.136565 \\
\bottomrule
\end{tabular}
\caption{Placement results proportion probabilities}
\end{table}

When averaged across samples, we get the following table:

\begin{table}
\begin{tabular}{lrrr}
\toprule
{} &  GuessesActive &  GuessesEx &  GuessesSamp \\
\midrule
0 &       0.130387 &   0.192563 &     0.156200 \\
1 &       0.102735 &   0.188372 &     0.105133 \\
2 &       0.187018 &   0.196050 &     0.160565 \\
\bottomrule
\end{tabular}
\caption{Average of proportion probabilities}
\end{table}

The probability of drawing either of the values closest to expectation, 3 or 4,
are 0.26 and 0.22 respectively. The average of all values within this table is
0.15, significantly lower than either expected value. Still, this is not enough
to reject the null hypothesis.

\subsection{t score}
In order to confidantly reject the null hypothesis, we will make use of the
T test.  The t test equation is as follows:
$$t =\frac{\overline{x}-\mu}{\hat{\sigma}_{\overline{x}}} = \frac{\overline{x}-\mu}{\frac{s}{\sqrt{N}}}$$
where s is the sample standard deviation, N is the
number of samples, overscore x is an individual samples mean/calculated value,
and mu is the population mean/expected value.

The idea is simple, take the diffrence between the observed mean of the sample
and the expected mean and normalize this value by the standard deviation of the
samples distribution. This results in some number of sample standard deviations
by which the observed sample distributions mean differs from expectation. The
fewer the number of sample the higher the margin of error in the t scores
estimate, with a margin of error of 0.05 given for estimates made with 30
samples. The critical thresholds for a two tailed t test are -1.96 and 1.96.
If the averaged values of the t scores falls outside these bounds then we can
reject the null hypothesis with a 5 percent margin of error. The standard
deviations and t scores for each of the samples follows in table 4.4:

\begin{table}
\begin{tabular}{lrrr}
\toprule
{} &  GuessesActive &  GuessesEx &  GuessesSamp \\
\midrule
First &       1.417745 &   1.301281 &     1.477611 \\
Second &       1.542365 &   1.530795 &     1.056199 \\
Third &       1.364633 &   1.325393 &     1.580787 \\
\bottomrule
\end{tabular}
\caption{Placement results standard deviations}
\end{table}


\begin{table}
\begin{tabular}{lrrrrrrrrr}
\toprule
algorithms & \multicolumn{3}{l}{GuessesActive} & \multicolumn{3}{l}{GuessesEx} & \multicolumn{3}{l}{GuessesSamp} \\
positions &             0 &          1 &          2 &          0 &          1 &          2 &           0 &          1 &          2 \\
\midrule
0  &    -10.408994 &   3.240168 &   7.133764 &  10.934820 &  -5.508733 &  -7.984048 &   -1.544874 &   3.181223 &  -1.333630 \\
1  &    -10.408994 &   3.240168 &   7.133764 &   6.834263 &  -5.508733 &  -1.996012 &    3.089747 &   3.181223 &  -5.334519 \\
2  &    -10.408994 &  -1.620084 &  11.414022 &  15.035378 &  -1.377183 & -19.960120 &   -6.179495 &   3.181223 &   2.667259 \\
3  &    -10.408994 &   8.100420 &   2.853506 &  10.934820 &  -1.377183 & -13.972084 &   -1.544874 &  -6.362445 &   6.668149 \\
4  &    -14.869991 &  12.960671 &   2.853506 &  19.135935 &  -5.508733 & -19.960120 &   -6.179495 &  -6.362445 &  10.669038 \\
5  &     -1.486999 &  -1.620084 &   2.853506 &   6.834263 &   2.754366 & -13.972084 &   -6.179495 &  -1.590611 &   6.668149 \\
6  &      2.973998 &  -1.620084 &  -1.426753 &   2.733705 &   2.754366 &  -7.984048 &   -6.179495 &  -1.590611 &   6.668149 \\
7  &     -5.947996 &  -1.620084 &   7.133764 &  15.035378 &  -5.508733 & -13.972084 &  -10.814116 &   7.953056 &   2.667259 \\
8  &    -10.408994 &  -1.620084 &  11.414022 &  -1.366853 &   6.885916 &  -7.984048 &   12.358990 &  -6.362445 &  -5.334519 \\
9  &    -14.869991 &   3.240168 &  11.414022 &  15.035378 &  -1.377183 & -19.960120 &   -1.544874 &  -1.590611 &   2.667259 \\
10 &    -14.869991 &  12.960671 &   2.853506 &  15.035378 &  -1.377183 & -19.960120 &   -1.544874 & -11.134279 &  10.669038 \\
11 &    -10.408994 &   8.100420 &   2.853506 &  15.035378 &  -5.508733 & -13.972084 &   -6.179495 &  -1.590611 &   6.668149 \\
12 &     -1.486999 &  -1.620084 &   2.853506 &   6.834263 &   2.754366 & -13.972084 &   -6.179495 &  -1.590611 &   6.668149 \\
13 &     -5.947996 &   8.100420 &  -1.426753 &  10.934820 &  -1.377183 & -13.972084 &   -6.179495 &  -6.362445 &  10.669038 \\
14 &     -5.947996 & -11.340587 &  15.694280 &   2.733705 &  11.017465 & -19.960120 &    3.089747 &  -1.590611 &  -1.333630 \\
15 &    -10.408994 &   8.100420 &   2.853506 &  10.934820 & -13.771832 &   3.992024 &   -1.544874 &   7.953056 &  -5.334519 \\
16 &    -10.408994 &   3.240168 &   7.133764 &  10.934820 &   2.754366 & -19.960120 &   -1.544874 &  -6.362445 &   6.668149 \\
17 &    -10.408994 &  -1.620084 &  11.414022 &  19.135935 &  -9.640282 & -13.972084 &  -10.814116 &  12.724890 &  -1.333630 \\
18 &    -10.408994 &   3.240168 &   7.133764 &  15.035378 &  -1.377183 & -19.960120 &   -6.179495 &  -1.590611 &   6.668149 \\
19 &     -5.947996 &   3.240168 &   2.853506 &  10.934820 &  -5.508733 &  -7.984048 &   -6.179495 &   3.181223 &   2.667259 \\
20 &    -10.408994 &  -6.480336 &  15.694280 &   2.733705 &  11.017465 & -19.960120 &    7.724369 &  -6.362445 &  -1.333630 \\
21 &     -1.486999 &  -1.620084 &   2.853506 &  -5.467410 &  15.149015 & -13.972084 &    7.724369 & -15.906113 &   6.668149 \\
22 &     -1.486999 &   3.240168 &  -1.426753 &  10.934820 &   2.754366 & -19.960120 &  -10.814116 &  -6.362445 &  14.669927 \\
23 &     -1.486999 &  -1.620084 &   2.853506 &   2.733705 &   2.754366 &  -7.984048 &   -1.544874 &  -1.590611 &   2.667259 \\
24 &     -5.947996 &  12.960671 &  -5.707011 &  15.035378 &  -1.377183 & -19.960120 &  -10.814116 & -11.134279 &  18.670816 \\
25 &    -10.408994 &  -1.620084 &  11.414022 &  10.934820 &  -5.508733 &  -7.984048 &   -1.544874 &   7.953056 &  -5.334519 \\
26 &     16.356990 &  -6.480336 &  -9.987269 &  -1.366853 &   6.885916 &  -7.984048 &  -15.448737 &  -1.590611 &  14.669927 \\
27 &    -14.869991 &   8.100420 &   7.133764 &  15.035378 &  -5.508733 & -13.972084 &   -1.544874 &  -1.590611 &   2.667259 \\
28 &    -10.408994 &  -6.480336 &  15.694280 &   2.733705 &   6.885916 & -13.972084 &    7.724369 &  -1.590611 &  -5.334519 \\
29 &     -5.947996 &  12.960671 &  -5.707011 &   2.733705 &  -1.377183 &  -1.996012 &    3.089747 & -11.134279 &   6.668149 \\
\bottomrule
\end{tabular}
\caption{Placement results t scores}
\end{table}

Table 4.5 contains these t scores averaged across samples:

\begin{table}
\begin{tabular}{lrrr}
\toprule
{} &  GuessesActive &  GuessesEx &  GuessesSamp \\
\midrule
0 &      -7.286296 &   2.268117 &     4.993635 \\
1 &       9.157912 &  -0.137718 &   -13.173679 \\
2 &      -2.780773 &  -2.067795 &     4.134252 \\
\bottomrule
\end{tabular}
\caption{Average of t scores}
\end{table}

Taking the average of the absolute value of each of these t scores yeilds
5.11. We can thus comfortably reject the null hypothesis.

%%Chapter 5
\chapter{Conclusion}
\label{Chapter5}
In this thesis I proposed that the no free lunch hypotheis might not apply to
meta learning algorithms. In order to test this hypothesis, I first built a
system to determine the accuracy of three meta learning strategies:
Exhaustive, Active, and Sampling. To use these strategies, a base of
datasets would first be randomly choosen from the collection of all available
datasets that had been gathered from the UCI Irvine data repository.
Each strategy would then be carried out on the metabase and make an
estimate as to what algorithm would result in the highest classification
accuracy. Each algorithm would make this guess for each dataset in the
collection of available datasets excluding the datasets within the current
metabase. A new metabase would then be choosen at random and the process would
repeated 9 more times, giving a number between 0 and 10 for how many times
each algorithm got the First, second, or third most correct guesses.
This process was repeated 30 times, resulting in 30 samples. $t$ test analysis
was then performed, giving an an average among the absolute values of each of
the position $t$ scores of 5.11, allowing us to reject the null
hypothesis at a 5 percent margin of error.

A flaw exist within this experiment that I would correct had I more time: I was
only able to obtain one set of datasets with 88 instances in it. As such there
is a possibility of bias in the data. This potential of bias can be mitigated
with the introduction of extra sets of datasets, with the ideal being a unique
set of datasets for each sample, but I unfortunately do not currently have the
time to obtain and transform any more data.


%----------------------------------------------------------------------------------------
%	THESIS CONTENT - APPENDICES
%----------------------------------------------------------------------------------------

\appendix % Cue to tell LaTeX that the following "chapters" are Appendices

% Include the appendices of the thesis as separate files from the Appendices folder
% Uncomment the lines as you write the Appendices

\include{Appendices/AppendixA}
%\include{Appendices/AppendixB}
%\include{Appendices/AppendixC}

%----------------------------------------------------------------------------------------
%	BIBLIOGRAPHY
%----------------------------------------------------------------------------------------

\begin{thebibliography}
%General
\bibitem(Murphy)
Kevin P. Murphy
\textit{Machine Learning: A probabilistic Perspective}

\bibitem{Wolpert}
  David H. Wolpert; William G. Macready
  \textit{No free Lunch Theorems for Optimization}
  IEEE Transactions on evolutionary computation vol 1 no.1 April 1997

%Metalearning
\bibitem{Lemke}
 Lemke, Christiane; Budka, Marcin
 \textit{Metalearning: a survey of trends and technologies} 2015

\bibitem{Bhatt}
 Bhatt, Thakkar; Ganatra, Bhatt
 \textit{Ranking of Classifiers based on dataset charactersistics using active meta learning} 2013

\bibitem{Leite}
 Leite, Rui; Brazdil, Pavel
\ltextit{Predicting Relative Performance of Classifiers from Samples}

%Active learning
\bibitem{Settles}
Settles, Burr
\textit{Active Learning Literature Survey}
 Computer Sciences Technical Report 1648. University of Wisconsin–Madison 2014

%svm
\bibitem{Vapnik}
\textit{support vector networks" Cortes, C. & Vapnik, V. Machine Learning } 1995
20: 273. https://doi.org/10.1023/A:1022627411411

%k-means
\bibitem{MacQueen}
Macquenn
\textit{Some methods for classification and analysis of multivariate observations}

\bibitem{Arthur}
Arthur,David; Vassilvitskii,Sergei
 \textit{k-means++: The advantages of careful seeding}

%Neural Networks
\bibitem{Rosenblatt}
Rosenblatt, F.
\textit{The Perceptron: A Probabilistic Model For Information Storage And Organization In The Brain}.%Paywalled

\bibitem{Scikit}
scikit-learn.org
\textit{https://scikit-learn.org/stable/modules/neural_networks_supervised.html}

\bibitem{Hornik}
Hornik, Kurt;
\textit{Approximation Capabilities of Multilayer Feedforward Networks}

\bibitem{Schmindhuber}
Schmindhuber, Jurgen
\textit{Deep learning in neural networks: An overview} 2014

\bibitem{Rumelhart}
Rumelhart, David E., Geoffrey E. Hinton, and Ronald J. Williams.
\textit{Learning representations by back-propataging errors} 1986

\end{thebibliography}

%----------------------------------------------------------------------------------------

\end{document}
