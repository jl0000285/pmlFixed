%Chapter 4
\chapter{Research Findings}
\label{Chapter4}
\section{Run results and Analysis tools}
These results are clearly not what we expect if the null hypothesis is true. Testing of this hypothesis can be done with the machinary of classical statistics, specifically by calculating
the exact probability of the results by working out the exact sampling distribution and by obtaining how likely these results are to come from the various appropriate distributions where
the performances are equal by running the results thru the machinary of the t-test. A description of each of these statistical methods follows.
\subsection{Exact Sampling Distribution}
The following discription closely follows what is outlines the procedure described in \cite{Cohen}. In it, the author asks you to imagine testing a coin to see whether or not it
is fair, flipping the coin 1,2,..N times. He then asks you to consider whether some proportion of heads is actually fair from 0/N, 1/N.., N/N heads. The propability that some proportion of heads
p = i/N is fair can be calculated exactly with the binomial distribution $$\frac{N!}{i!(N-i)!}r^{i}(1-r)^{N-i}$$. This situation is analogous to the number of first, second, or third place finishes
some meta-algorithm obtained in this thesis experiment. The probabilty of proportion for each of the meta-learning algorithms can be seen in table

\begin{table}[]
\begin{tabular}{lll}
Meta Algorithm & Proportion of Positions & Probability \\
GuessesSamp    & 8/10 first place        &             \\
GuessesSamp    & 2/10 second place       &             \\
GuessesSamp    & 0/10 third place        &             \\
GuessesActive  & 2/10 first place        &             \\
GuessesActive  & 5/10 second place       &             \\
GuessesActive  & 3/10 third place        &             \\
GuessesEx      & 0/10 first place        &             \\
GuessesEx      & 3/10 second place       &             \\
GuessesEx      & 7/10 third place        &
\end{tabular}
\end{table}

T test equation where s is the sample standard deviation and N is the number of samples, overscore x is an individual samples mean/calculated value, mu is the
population mean/expected value
$$t =\frac{\overline{x}-\mu}{\hat{\sigma}_{\overline{x}}} = \frac{\overline{x}-\mu}{\frac{s}{\sqrt{N}}}$$
